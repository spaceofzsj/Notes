\documentclass[10pt]{article}
\usepackage{NotesTeXV3}
\usepackage{orcidlink}
\usepackage{tensor}
\usepackage{cleveref}
\usepackage{thmtools}


\let\oldsection\section
\def\section{%
  \setcounter{sidenote}{1}%
  \oldsection
}

\newcommand{\me}{\mathrm{e}}
\DeclareMathOperator{\im}{im}


\begin{document}
\title{Topology and Geometry}
\subheader{Notes}
\author{Shangjie Zhou\orcidlink{0000-0001-9576-5011}}
\affiliation{School of Physics and Technology, Wuhan University}
\emailAdd{sjzhou@whu.edu.cn}
\abstract{\textit{Last updated on: \today}\\\url{https://github.com/spaceofzsj/notes}}

\maketitle

\listoftheorems[title={List of Definitions},ignoreall,onlynamed={definition},swapnumber]
\listoftheorems[ignoreall,onlynamed={theorem},swapnumber]
\clearpage

\section{Mathematical Preliminaries}
\subsection{Maps}
\subsubsection{Definitions}
\subsubsection{Equivalence Relation and Equivalence Class}
\begin{definition}[Equivalence relation]
    An equivalence relation $\sim$ is a relation which satisfies the following requirements
    \begin{enumerate}
        \item \textit{reflective}: $a\sim a$.
        \item \textit{symmetric}: If $a\sim b$, then $b\sim a$.
        \item \textit{transitive}: If $a\sim b$ and $b\sim c$, then $a\sim c$.
    \end{enumerate}
\end{definition}
\subsection{Vector Spaces}
\subsubsection{Vectors and Vector Spaces}
\begin{definition}[Vector space]
    A \textbf{vector space} (or a \textbf{linear space}) $V$ over a field $K$ is a set in which two operations, addition and multiplication by an element of $K$ (called a \textbf{scalar}),
    are defined (In this notes we are mainly interested in $K=\mathbb{R}$ and $\mathbb{C}$).
    The elements of $V$ (called \textbf{vectors}) satisfy the following axioms:
    \begin{enumerate}
        \item $\vb*{u}+\vb*{n}=\vb*{v}+\vb*{u}$.
        \item $(\vb*{u}+\vb*{v})+\vb*{w}=\vb*{u}+(\vb*{v}+\vb*{w})$.
        \item There exists a zero vector $\vb*{0}$ such that $\vb*{v}+\vb*{0}=\vb*{v}$.
        \item For any $\vb*{u}$, there exists $-\vb*{u}$, such that $\vb*{u}+(-\vb*{u})=0$.
        \item $c(\vb*{u}+\vb*{v})+c\vb*{u}+c\vb*{v}$.
        \item $(cd)\vb*{u}=c(d\vb*{u})$.
        \item $1 \vb*{u}=\vb*{u}$.
    \end{enumerate}
    Here $\vb*{u}, \vb*{v}, \vb*{w}$ and $c,d\in K$ and $1$ is the unit element of $K$.
\end{definition}
\subsubsection{Linear Maps, Images and Kernels}
\begin{theorem}
    If $f:V\to W$ is a linear map, then
    \begin{align}
        \dim{V}=\dim(\ker{f})+\dim(\im f).
    \end{align}
\end{theorem}
\begin{proof}
    to be added
\end{proof}

\subsubsection{Dual Vector Space}

\subsubsection{Inner Product and Adjoint}
\begin{theorem}[Toy index theorem]
    Let $V$ and $W$ be finite-dimensional vector spaces over a field $K$ and let $f:V\to W$ be a linear map.
    Then
    \begin{align}
        \dim\ker f-\dim\ker\tilde{f}=\dim V-\dim W.
    \end{align}
\end{theorem}
\begin{proof}
    to be added
\end{proof}

\subsubsection{Tensors}

\subsection{Topological Spaces}
\subsubsection{Definitions}
\subsection{Homeomorphisms and Topological Invariants}
\subsubsection{Homeomorphisms}
\begin{intu}
    The purpose of topology is to classify spaces.
\end{intu}

For example, in elementary geometry, the equivalence of figures is given by congruence, which turns out to be too stringent for our purpose.
In topolgy, we define two figures to be equivalent if it is possible to deform one figure into the other by \textit{continuous deformation}.
\begin{definition}[Homeomorphism]\label{homeomorphism}
    Let $X_1$ and $X_2$ be topological spaces.
    A map $f : X_{1}\to X_{2}$ is a \textbf{homeomorphism} if it is continuous and has an inverse $f^{-1}:X_{2}\to X_{1}$ which is also continuous.
    If there exists a homeomorphism between $X_1$ and $X_2$ , $X_1$ is said to be \textbf{homeomorphic} to $X_2$ and vice versa.
\end{definition}

A homeomorphism is an equivalence relation and the proof is:
\begin{proof}
    \begin{itemize}
        \item Reflectivity: simply choose $f=\text{id}_X$.
        \item Symmetry: by definition, if $f : X_1 \to X_2$ is a homeomorphism, so is $f^{-1} : X_2 \to X_1$. (Using $(f^{-1})^{-1}=f$)
        \item Transitivity: if $f : X_1 \to X_2$ and $g : X_2 \to X_3$ are homeomorphisms, so is $g\circ f : X_1\rightarrow X_3$.
    \end{itemize}
\end{proof}

Intuitively speaking, two topological spaces are homeomorphic to each other if we can deform one into the other continuously, that is, without tearing them apart or pasting.
\begin{example}
    An open disc $D^2= \left\{ (x,y)\in\mathbb{R}^{2}|x^2+y^2<1 \right\}$ is homeomorphic to $\mathbb{R}^2$ when we choose
    \begin{align}
        f(x,y)=\left(\frac{x}{\sqrt{1-x^2-y^2}},\frac{y}{1-x^2-y^2}\right)
    \end{align}
\end{example}

\subsubsection{Topological Invariants}
\textbf{Topological invariants} are quantities to characterize equivalence clsses of homeomorphism.

If two spaces are homeomorphic to each other,they have the same "topological invariants".
And if two spaces have different "topological invariants",they are not homeomorphic to each other.

\subsubsection{Homotomy type}
\begin{intu}
The condition of homeomorphism is too restrict, we will see it would be useful to define a new equivalent class, namely of the \textit{same homotopy type}.
\end{intu}

Just relax the condition in \cref{homeomorphism} so that the continuous functions $f$ and $g$ need not have inverses.

\subsubsection{Euler Characteristic: An Example}
In this section, We focus on $\mathbb{R}^3$.
\begin{definition}[Euler characteristic]
    Let $X$ be a subset of $\mathbb{R}^3$, which is homeomorphic to a polyderon $K$.
    Then the \textbf{Euler characteristic} $\chi(X)$ of $X$ is defined by
    \begin{align}
        \chi(X)= & (\text{number of verticies in}\ K)-(\text{number of edges in}\ K)\notag \\
                 & +(\text{number of faces in}\ K)
    \end{align}
\end{definition}

\begin{theorem}[Poincar{\'e}-Alexander]
    The Euler characteristic $\chi(X)$ is idependent of the polyhedron $K$ as long as $K$ is homeomorphic to $X$.
\end{theorem}

The \textit{connected sum} $X \sharp Y$ of two surfaces $X$ and $Y$ is a surface obtained by removing a small disc from each $X$ and $Y$ and connecting the resulting holes with a cylinder.

It is obvious that $ S^{2} \sharp X=X $ and $T^{2}\sharp T^{2}$.
An immediate generalization is
\begin{align}
    \underbrace{T^{2}\sharp T^{2}\sharp\cdots\sharp T^{2}}_{g\ \text{factors}}=\Sigma_g.
\end{align}

\begin{theorem}
    Let $X$ and $Y$ be two surfaces.
    Then the Euler characteristic of the connected sum $X\sharp Y$ is given by
    \begin{align}
        \chi(X\sharp Y)=\chi(X)+\chi(Y)-2.
    \end{align}
\end{theorem}
\begin{proof}
    to be added
\end{proof}

\begin{theorem}
    Let $X$ and $Y$ be two figures in $\mathbb{R}^3$.
    If $X$ is homeomorphic to $Y$, then $\chi(X)=\chi(Y)$.
    In other words, if $\chi(X)\neq\chi(Y)$, $X$ cannot be homeomorphic to $Y$.
\end{theorem}
While compactness and connectedness are geometrical properties of a figure or a space $X$, the Euler characteristic is a integer $\chi(X)\in\mathbb{Z}$, which is algebraic.
It implies that the Euler characteristic can relate geometry to algebra\sidenote{We may compute the Euler characteristic of a smooth surface with \textit{Gauss-Bonnet theorem}.}.

\clearpage
\section{Homology Groups}
\subsection{Abelian Groups}
\subsubsection{Elementary Group Theory}

\begin{theorem}[Fundamental Theorem of Homomorphism]
    Let $f:G_1\to G_2$ be a homomorphism. Then
    \begin{align}
        G_1/\ker f\cong \im f.
    \end{align}
\end{theorem}
\begin{proof}
    to be added
\end{proof}

\subsubsection{Finitely Generated Abelian Groups and Free Abelian Groups}
\begin{definition}[Free abelian group]
    If $G$ is finitely generated by $r$ \textit{linearly independent} elements, $G$ is called a \textbf{free Abelian group} of \textbf{rank} $r$.
\end{definition}

\subsubsection{Cyclic Groups}
    
\begin{theorem}[Fundamental theorem of finitely generated Abelian groups]
    Let $G$ be a finitely generated Abelian group (not necessarily free) with $m$ generators. 
    Then $G$ is isomorphic to the direct sum of cyclic groups,
    \begin{align}
        G\cong\mathbb{Z}\oplus\cdots\oplus\mathbb{Z}\oplus\mathbb{Z}_{k_1}\oplus\cdots\oplus\mathbb{Z}_{k_p}
    \end{align}
    where $m=r+p$. 
    The number $r$ is called the \textbf{rank} of $G$.
\end{theorem}
\begin{proof}
    to be added
\end{proof}


\subsection{Simplexes and Simplicial Complexes}
\subsection{Homology Groups of Simplicial Complexes}
\subsection{General Properties of Homology Groups}

\clearpage
\section{Homotopy Groups}
\subsection{Fundamental Groups}
\subsection{General Properties of Fundamental Groups}
\subsection{Examples of Fundamental Groups}
\subsection{Fundamental Groups of Polyhedra}
\subsection{Higher homotopy groups}
\subsection{General Properties of Higher Homotopy Groups}
\subsection{Examples of Higher Homotopy Groups}
\subsection{Orders in Condensed Matter Systems}
\subsubsection{Order Parameter}
\textbf{Spontaneous symmetry breakdown}: Assume the the hamiltonian of a system $H$ is invariant under a certain symmetry operation.
The ground state of the system need not preserve the symmetry of $H$.
\begin{example}[Heisenberg hamiltonian]
    The Heisenberg hamiltonian is
    \begin{align}
        H=-J\sum_{(i,j)}\vec{S}_i\cdot\vec{S}_j+\vec{h}\cdot\sum_i \vec{S}_i.
    \end{align}
    Note that we are talking about quantum mechanics so $H$ here is an operator.

    The $\vec{S}_i$ are ferromagnetic Heisenberg spins and $J=\text{const}>0$.
    The summation is over the pair of the nearest-neighbour sites $(i,j)$ and $\vec{h}$ is the uniform external magnetic field.

    The \textit{partition function} is $Z\equiv\Tr\left(\text{e}^{-\beta H}\right)=\sum_i \me^{-\beta E_i}$ where $\beta=1/T$ is the inverse temperature and $E_i$ are eigenvalues of $H$.
    The \textit{free energy} is defined by $\exp(-\beta F)=Z$.
    The average magnetization per spin is defined as:
    \begin{align}
        \vec{m}\equiv\frac{1}{N}\sum_{i}\langle\vec{S}_i\rangle=\frac{1}{N\beta}\frac{\partial F}{\partial \vec{h}},
    \end{align}
    where $\langle\dots\rangle\equiv\Tr(\dots \text{e}^{-\beta H})/Z$ can be interpreted to be the expectation value.

    In the limit $\vec{h}\rightarrow 0$ ($H\rightarrow -J\sum_{(i,j)}\vec{S}_i\cdot\vec{S}_j$), $H$ is invariant under $\text{SO}(3)$ but $\vec{m}$ doesn't vanish because of the ferromagnetism.
    It is said that the system exhibits \textbf{spontaneous magnetization}.

    In this example, the vector $\vec{m}$ is called the \textbf{order parameter} describing the phase transition between the ordered state and the disordered state.
\end{example}
The free energy of a system is defined by
\begin{align}
    F=\langle H\rangle-TS,
\end{align}
$S$ is entropy and equilibrium is obtained by minimizing $F$.

When temperature is low, $F\approx\langle H\rangle$.
All $S_i$ align in the same direction.
When temperature is high, $F\approx-TS$.
All $S_i$ are totally random.

In the Heisenberg Hamiltonian example, when the temperature is uniform, $\abs{\vec{m}(\vec{x})}$ is constant so $\vec{m}$ can be characterized by polar coordinates $(\theta,\phi)$ on $S^2$.
Now we have a map $\vec{m}: \vec{x}\to\vec{m}(\vec{x})$ or $\mathbb{R}^3\rightarrow S^2$.

The excited states of this system depend on this map $\vec{m}$ (in other systems, the order parameter may be not a vector).
Now we make a generalization to all systems that these kinds of excitation depends on the topological arguments are called \textbf{topological excitations}.

\subsubsection{Superfluid \texorpdfstring{{}$^4\text{He}$}{} and superconductors}
The order parameter of  $^4\text{He}$ is the expectation value:
\begin{align}
    \expval{\phi(\vec{x})}=\Psi(\vec{x})=\Delta_0 (\vec{x})\me^{i\alpha(\vec{x})}.
\end{align}

In the BCS theory of superconductors, the order parameter is given by:
\begin{align}
    \psi_{\alpha\beta}\equiv\expval{\psi_\alpha(\vec{x})\psi_\beta(\vec{x})}.
\end{align}
We can see that the order parameter may even not be a scalar.

\subsubsection{General Consideration}
The order parameter of a 3D superconductor is $\psi(x)=\Delta_0(\vec{x})\text{e}^{i\phi(\vec{x})}$.
When a homogeneous system is under uniform external conditions, $\Delta_0$ is a constant.
So $\psi(x)\cong\text{U(1)}$.

We call the collection of order parameter values the \textbf{order parameter space} $M$.
For superconductor, $M=\text{U(1)}$ and for nematic liquid crystal $M=\mathbb{R}P^2$.

There are cases that the order parameter takes value outside $M$ (inhomogeneous state), but we still assume order parameters take value in $M$ when the scale of the system is large enough.

However, there are also some points, lines or surfaces where the order parameters may not be defined uniquely.
We call them \textbf{point defects (monopoles), line defects (vortices) and surface defects (domain walls)}.

Consider a system, the order parameter is a classical field $\psi(x)$ and it can be regarded as a map $\psi:X\to M$.

Imagine there is a line defect.
We enclose it by a circle $S^1$ and pick a reference point $x_0$ on it.
Thus, we just defined a loop $S^1\to M$.
Recall from the homotopy group, we can now classify this loop according to the homotopy class they belong to.

For example, for superconductor $M=\text{U(1)}$ and $\pi_1\left(\text{U(1)},x_0\right)\cong\mathbb{Z}$.
So it's possible to assign an integer to each loop and we call it the \textbf{winding number} of the loop.

We can also generlize to other kinds of defects.
Just pick a sphere $S^n$ which surrounds the defect and classify this sphere by homotopy classes or $\pi_n(M,x_0)$.
\begin{remark}
    Assume there is an $m$-dimensional defect $A$ in $X$.
    If we can find an $n$-dimensional sphere $S^n$ which crosses the defect when continuously shrink to a point $S^0$ in any way, then we say this sphere surrounds $A$.
\end{remark}

In general, an $m$-dimensional defect in a $d$-dimensional medium is classified by the homotopy group $\pi_n(M,x_0)$ where:
\begin{align}
    n=d-m-1
\end{align}

\subsection{Defects in Nematic Liquid Crystals}
\subsubsection{Order Parameter of Nematic Liquid Crystals}
An example of nematic liquid crystal is \textit{octyloxy-cyanobiphenyl} whose molecule looks like a rod and the order parameter of it is \textbf{director} which is given by the average direction of the "rod".

The director has an inversion symmetry which means we can not distinguish the directors $\vec{n}=\rightarrow$ and $-\vec{n}=\leftarrow$, so the order parameter lives in the \textbf{projective plane} $\mathbb{R}P^2$.

The director in general depends on the position $\vec{r}$.
Then we may define a map $f:\mathbb{R}^3\to\mathbb{R}P^2$ and this map is called the \textbf{texture}.

\subsubsection{Line Defects in Nematic Liquid Crystals}
From the fact that $\pi_1(\mathbb{R}P^2)\cong\mathbb{Z}_2=\{0,1\}$, there exist two kind of line defect in the nematic liquid crystals;
one can be continuously deformed into a uniform configuration while the other cannot.
\begin{remark}[continuous deformation]
    Take $M=\mathbb{R}P^2$ as an example.
    Denote the initial distribution is A$(\vec{r})$.
    We say $A(\vec{r})$ can be continuously deformed into if there exists an \textit{continuous map} $f:\mathbb{R}^3\times I\to\mathbb{R}P^2$ where $f$ satisfies:
    \begin{enumerate}
        \item $f(\vec{r},0)=A(\vec{r})$,
        \item $f(\vec{r},1)=B(\vec{r})$.
    \end{enumerate}
\end{remark}

\subsubsection{Point Defects in Nematic Liquid Crystals}
A line defect and a point defect may be (\textit{when the line defect is "closed"})conbined into a \textbf{ring defect}.

\subsubsection{Higher Dimensional Texture}
After choosing a \textit{standard director} $\vec{n}_0$, a director $\vec{n}$ at any position can be characterized by a 3-dimensional rotation $R(\vec{e},\alpha)\in\text{SO(3)}$: $\vec{n}=R(\vec{e},\alpha)\vec{n}_0$.

An arbitrary rotation in $\mathbb{R}^3$ is specified by a unit vector $\vec{e}$, around which the rotation is carried out, and the rotation angle $\alpha$.
It is possible to assign a "vector" $\vec{\Omega}=\alpha \vec{e}$ to the rotation.
It is not exactly a vector since $\Omega =\pi\vec{e}$ and $-\Omega=-\pi\vec{e}$ are the same rotation and hence should be identified.
Therefore, $\vec{\Omega}$ belongs to the real projective space $\mathbb{R}P^3$.

\subsection{Textures in Superfluid \texorpdfstring{{}$^3\text{He-A}$}{}}
\subsubsection{Superfluid \texorpdfstring{{}$^3\text{He-A}$}{}}
Order parameter of $^4$He-A is
\begin{align}
    A_{\mu i}=d_\mu (\vec{\Delta}_1+i\vec{\Delta}_2)_i
\end{align}
where $\vec{d}$ is a unit vector along which the spin projection of the Cooper pair vanishes and $\left(\vec{\Delta}_1,\vec{\Delta}_2\right)$ is a pair of orthonormal unit vectors.

If we define $\vec{l}\equiv\vec{\Delta}_1\times\vec{\Delta}_2$, the triad $(\vec{\Delta}_1,\vec{\Delta}_2,\vec{l})$ forms an orthonormal frame at each point of the medium.
Since any orthonormal frame can be obtained from a standard orthonormal frame $(\vec{e}_1,\vec{e}_2,\vec{e}_3)$ by an application of a three-dimensional rotation matrix.
Thus, the order parameter is $S^2\times\text{SO(3)}$.

For simplicity, we ingore the variation of $\hat{d}$.
So the order parameter has the form:
\begin{align}
    A_i=\Delta_0\left(\hat{\Delta}_1+\hat{\Delta}_2\right)_i.
\end{align}
The order parameter have defined a map $\psi:X\to\text{SO(3)}$.
$\psi$ is called the texture of $^3\text{He}$.
The relevant homotopy groups for classifying defects in superfluid $^3\text{He-A}$ are $\pi_n(\text{SO(3)})$.

\subsubsection{Line Defects and Non-singular Vortices in \texorpdfstring{$^3\text{He-A}$}{}}
From the fact $\text{SO(3)}\cong\mathbb{R}P^3$, we have $\pi_1(\text{SO(3)})\cong\pi_1(\mathbb{R}P^3)\cong\mathbb{Z}_2\cong\{0,1\}$.

Textures which belong to class 0 can be continuously deformed into the uniform configuration.
Configurations in class 1 are called \textbf{disgyrations}.

\subsubsection{Shankar monopole in \texorpdfstring{$^3\text{He-A}$}{}}
If we specify an element in $\text{SO(3)}$ on each position of the space and denote the distribution by $\vec{\Omega}\left(\vec{x}\right)$ \sidenote{Remember that we can specify a rotation by an unit vector $\vec{n}$ and the rotation angle $\alpha$.}, then we get a texture.

Shankar proposed a texture:
\begin{align}
    \vec{\Omega}(\vec{r})=\frac{\vec{r}}{r}\cdot f(r),
\end{align}
where
\begin{align}
    f(\vec{r})=\begin{cases}
        2\pi & r=0       \\\\
        0    & r=\infty.
    \end{cases}
\end{align}
This texture is called \textbf{Shankar monopole}.

\clearpage
\section{Manifolds}
\subsection{Manifolds}
\subsection{The Calculus on Manifolds}
\subsection{Flows and Lie Derivatives}
\subsection{Differential Forms}
\subsection{Integration of Differential Forms}
\subsubsection{Orientation}
An integration of a differential form over a manifold $M$ is defined only when $M$ is \textit{orientable}.

Suppose there is a point $p\in M$ and $p$ belongs to two (or more) different charts $U_i$ with local coordinates $x^\mu$ and $U_j$ with local coordinates $y^\alpha$.
Thus, $T_p M$ is spanned by either $\{e_\mu\}=\{\frac{\partial}{\partial x^\mu}\}$ or $\{\tilde{e_\alpha}\}=\{\frac{\partial}{\partial y^\alpha}\}$.
The basis transforms as:
\begin{align}
    \tilde{e}_\alpha=\left(\frac{\partial x^\mu}{\partial y^\alpha}\right)e_{\mu}.
\end{align}
If $J\equiv\det(\frac{\partial x^\mu}{\partial y^\alpha})>0$ for $\forall p\in U_i \cap U_j$, then $\{e_\mu\}$ and $\{\tilde{e}_\alpha\}$ are said to define the \textit{same orientation}.
If $J<0$ for $\forall p\in U_i \cap U_j$, then $\{e_\mu\}$ and $\{\tilde{e}_\alpha\}$ are said to define the \textit{opposite orientation}.
\begin{definition}
    Let $M$ be a connected manifold coverd by $\{U_i\}$.
    The manifold $M$ is \textbf{orientable} if, for any overlapping charts $U_i$ and $U_j$, there exist local coordinates $\{x^\mu\}$ for $U_i$ and $\{y^\alpha\}$ for $U_j$ such that $J=\det(\frac{\partial x^\mu}{\partial y^\alpha})>0$.
\end{definition}
If $M$ is non-orientable, $J$ cannot be positive in all intersections of charts.
If an $m$-dimensional manifold $M$ is orientable, there exists an $m$-form $\omega$ which vanishes nowhere.
This $m$-form $\omega$ is called a \textbf{volume element}.
Two volume elements $\omega$ and $\omega'$ are said to be equivalent if there exists a strictly positive function $h \in \mathcal{F}(M)$ such that $\omega=h\omega'$.
Any orientable manifold admits two inequivalent orientations.
\subsubsection{Integration of Forms}
If there is a function $f$ on manifold $M$, we first define the integration of an $m$-form $f\omega$ on a chart by
\begin{align}
    \int_{U_i}f\omega\equiv\int_{\phi(U_i)}f\left(\phi_{i}^{-1}(x)\right)h\left(\phi_{i}^{-1}(x)\right)\dd{x^1} \dd{x^2}\dots \dd{x^m}.
\end{align}
\begin{definition}
    Take an open covering $\{U_i\}$ of $M$ such that each point of $M$ is covered with a finite number of $U_i$.
    If a family of differentiable functions $\epsilon_i(p)$ satisfies
    \begin{enumerate}
        \item $0\le\epsilon_i(p)\le1$
        \item $\epsilon_i(p)=0$ if $p\notin U_i$
        \item $\epsilon_1(p)+\epsilon_2(p)+\dots=1$ for any point $p\in M$
    \end{enumerate}
    the family $\{\epsilon(p)\}$ is called a \textbf{partition of unity} subordinate to the covering $\{U_i\}$.
\end{definition}
\begin{remark}
    $\epsilon_i$s are globally defined on the manifold $M$.
\end{remark}
So we have:
\begin{align}
    f(p)=\sum_{i}f(p)\epsilon_i(p)=\sum_i f_i(p),
\end{align}
where $f_i(p)\equiv f(p)\epsilon_i(p)$.
Note that $f_i(p)=0$ when $p\notin U_i$.
Now, we can define the integral of $f$ on the entire $M$ by
\begin{align}
    \int_M f\omega\equiv\sum_i \int_{U_i}f_i \omega.
\end{align}
\begin{remark}
    Different atlas give the same result of the integral.
\end{remark}
\begin{proof}
    Assume there are two (or more) different atlas $\{(U_i,\phi_i)\}$ with partition of unity $\epsilon_i$ and $\{(V_j,\psi_j)\}$ with partition of unity $\epsilon_j'$ defined on $M$, then
    \begin{align}
        \sum_i\int_{U_i}f_i\omega & =\sum_i\int_{\phi(U_i)}\epsilon_i\left(\phi_{i}^{-1}(x)\right) f\left(\phi_{i}^{-1}(x)\right)h\left(\phi_{i}^{-1}(x)\right)\widetilde{\dd{x}}\notag                                                        \\
                                  & =\sum_j \sum_i\int_{\phi(U_i)}\epsilon_j'\left(\phi_{i}^{-1}(x)\right)\epsilon_i\left(\phi_{i}^{-1}(x)\right) f\left(\phi_{i}^{-1}(x)\right)h\left(\phi_{i}^{-1}(x)\right)\widetilde{\dd{x}}\notag         \\
                                  & =\sum_j \sum_i\int_{\phi(U_i\cap V_j)}\epsilon_j'\left(\phi_{j}^{-1}(x)\right)\epsilon_i\left(\phi_{i}^{-1}(x)\right) f\left(\phi_{i}^{-1}(x)\right)h\left(\phi_{i}^{-1}(x)\right)\widetilde{\dd{x}}\notag \\
                                  & =\sum_j \sum_i\int_{\psi(U_i\cap V_j)}\epsilon_j'\left(\psi_{j}^{-1}(x)\right)\epsilon_i\left(\psi_{i}^{-1}(x)\right) f\left(\psi_{i}^{-1}(x)\right)g\left(\psi_{i}^{-1}(x)\right)\widetilde{\dd{y}}\notag \\
                                  & =\sum_j \sum_i\int_{\psi(V_j)}\epsilon_j'\left(\psi_{j}^{-1}(x)\right)\epsilon_i\left(\psi_{i}^{-1}(x)\right) f\left(\psi_{i}^{-1}(x)\right)g\left(\psi_{i}^{-1}(x)\right)\widetilde{\dd{y}}\notag         \\
                                  & =\sum_j \int_{\psi(V_j)}\epsilon_j'\left(\psi_{j}^{-1}(x)\right) f\left(\psi_{i}^{-1}(x)\right)g\left(\psi_{i}^{-1}(x)\right)\widetilde{\dd{y}}\notag                                                      \\
                                  & =\sum_j \int_{V_j}f_j'\omega,\notag
    \end{align}
    where $\widetilde{\dd{x}}\equiv\dd{x^1} \dd{x^2} \dots \dd{x^m}$ and $\widetilde{\dd{y}}\equiv\dd{y^1} \dd{y^2} \dots \dd{y^m}$.
\end{proof}
\subsection{Lie Groups and Lie Algebras}
\subsection{The Action of Lie Groups on Manifolds}

\clearpage
\section{de Rham Cohomology Groups}
\subsection{Stokes' Theorem}
\subsection{de Rham Cohomology Groups}
\subsection{Poincar{\'e}'s lemma}
\subsection{Structure of de Rham Cohomology Groups}

\clearpage
\section{Riemannian Geometry}
\subsection{Riemannian Manifolds and Pseudo-Riemannian Manifolds}
\subsubsection{Metric Tensors}

\begin{definition}[Riemannian metric]
    Let $M$ be a differentiable manifold.
    A \textbf{Riemannian metric} $g$ on $M$ is a type $(0, 2)$ tensor field on $M$ which satisfies the following axioms at each point $p \in M$:
    \begin{enumerate}
        \item $g_p(U,V)=g_p(V,U)$
        \item $g_p(U,U)\ge0$, where the equality holds only when $U=0$,
    \end{enumerate}
    where $U,V\in T_p M$ and $g_p\equiv\eval{g}_p$.
\end{definition}

A tensor field $g$ of type $(0, 2)$ is a \textbf{pseudo-Riemannian metric} if it satisfies:
\begin{enumerate}
    \item $g_p(U,V)=g_p(V,U)$
    \item if $g_p(U,V)=0$ for any $U\in T_p M$, then $V=0$.
\end{enumerate}

If there exists a metric $g$, we can define the inner product between two vectors $U, V\in T_pM$ to be $g_p(U,V)$.
It can be seen that $g(U,\cdot)$ is a map $T_pM\to \mathbb{R}$ by $V\mapsto g_p(U,V)$, which means that it can be identified with a one-form $\omega_U\in T_p^* M$.
Similarly, $\omega\in T^*_p M$ induces $V_\omega\in T_pM$ by $\langle \omega,U\rangle=g(V_\omega,U)$.
Thus, the metric $g_p$ gives rise to an isomorphism between $T_pM$ and $T^*_pM$.

Let $(U,\phi)$ be a chart in $M$ and $\{x^\mu\}$ the coordinates.
We can expand $g_p$ as:
\begin{align}
    g_p=g_{\mu\nu}(p)\dd x^\mu\otimes\dd x^\nu
\end{align}
and
\begin{align}
    g_{\mu\nu}(p)=g_p\left(\frac{\partial}{\partial x^\mu},\frac{\partial}{\partial x^\nu}\right) \quad (p\in M).
\end{align}

Subscript in $g_p$ is usually omitted.
It is common to regard $g_{\mu\nu}$ as a matrix.
Since $g_{\mu\nu}$ has maximal rank, we can define its inverse $g^{\mu\nu}$ by $g_{\mu\nu}g^{\nu\lambda}=g^{\lambda\nu}g_{\nu\mu}=\delta^\lambda_\mu$.

We define $g\equiv\det(g_{\mu\nu})$.
Clearly, $\det(g^{\mu\nu})=g^{-1}$.

The isomorphism between $T_p M$ and $T^*_p M$ is now expressed as
\begin{align}
    \omega_\mu=g_{\mu\nu},\quad U^\mu=g^{\mu\nu}\omega_\nu.
\end{align}

It's a convention to denote a metric in the form of an infinitestimal displacement
\begin{align}
    \dd s^2=g(\dd x^\mu\frac{\partial}{\partial x^\mu},\dd x^\nu\frac{\partial}{\partial x^\nu})=g_{\mu\nu}\dd x^\mu\dd x^\nu.
\end{align}

Since $g^{\mu\nu}$ is a symmetric matrix, the eigenvalues are real.
If $g$ is Riemannian, all the eigenvalues are strictly positive and if $g$ is pseudo-Riemannian, some of them may be negative.
If there are $i$ positive and $j$ negative eigenvalues, the pair $(i, j)$ is called the index of the metric.
If $j = 1$, the metric is called a Lorentz metric.

We can always diagonize and reduce the matrix element of $g$ to $\pm1$ by a change of basis.
After this procedure, if we start with a
\begin{itemize}
    \item Riemannian metric, we end up with the \textbf{Euclidean metric} $\delta=\text{diag}(1,\dots,1)$.
    \item Lorentz metric, we end up with the \textbf{Minkowski metric} $\eta=\text{diag}(-1,1,\dots,1)$.
\end{itemize}

If $(M,g)$ is Lorentzian, the elements of $T_p M$ can be devided into
\begin{enumerate}
    \item $g(U,U)>0\to U$ is \textbf{spacelike},
    \item $g(U,U)=0\to U$ is \textbf{lightlike}(or \textbf{null}),
    \item $g(U,U)<0\to U$ is \textbf{timelike}.
\end{enumerate}

If a smooth manifold $M$ admits a Riemannian metric $g$, the pair $(M, g)$ is called a \textbf{Riemannian manifold}.
If $g$ is a pseudo-Riemannian metric, $(M,g)$ is called a \textbf{pseudo-Riemannian manifold}.
If $g$ is Lorentzian, $(M, g)$ is called a \textbf{Lorentz manifold}.
\subsubsection{Induced Metric}
Let $M$ be an $m$-dimensional submanifold of an $n$-dimensional Riemanian manifold $N$ with the metric $g_N$.
If $f:M\to N$ is the embedding which induces the submanifold structure of $M$, the pullback map $f^*$ induces the natural metric $g_M=f^* g_N$ on $M$ (\textbf{induced metric}).
The components of $g_M$ are given by
\begin{align}
    g_{M\mu\nu}(x)=g_{N\alpha\beta}(f(x))\frac{\partial f^\alpha}{\partial x^\mu}\frac{\partial f^\beta}{\partial x^\nu}
\end{align}
where $f^\alpha$ denote the coordinates of $f(x)$.

\subsection{Parallel Transport, Connection and Covariant Derivative}
\subsubsection{Heuristic Introduction}
We want to compare vectors defined on different points on the manifold.
\begin{intu}
    Before we compare two vectors defined at different points, we have to "move" a vector defined at one place to another without "changing" it (\textbf{parallel transport}).
So it's necessary to specify the rules of the parallel transport, which requires an extra structure called \textbf{connection}.
\end{intu}
Let $\widetilde{V} \vert_{x+\Delta x}$ denote a vector $V|_{x}$ parallel transported $x+\Delta x$.
Heuristically, We demand that the components satisfy
\begin{align}
    \widetilde{V}^\mu(x+\Delta x)-V^\mu(x) & \propto\Delta x                                               \\
    \widetilde{(V^\mu+W^\mu)}(x+\Delta x)  & =\widetilde{V}^\mu(x+\Delta x)+\widetilde{W}^\mu(x+\Delta x).
\end{align}
These conditions are satisfied if we take
\begin{align}
    \widetilde{V}^\mu(x+\Delta x)=V^\mu(x)-V^\lambda(x)\tensor{\Gamma}{^{\mu}_{\nu\lambda}}\Delta x^\nu
\end{align}
There are many distinct rules of parallel transport possible, one for each choice of $\Gamma$.

If the manifold is endowed with a metric, there exists a preferred choice of $\Gamma$, called the \textbf{Levi-Civita connection}.
Levi-Civita connection satisfies
\begin{enumerate}
    \item $\tensor{\Gamma}{^{\lambda}_{\mu\nu}} = \tensor{\Gamma}{^{\lambda}_{\nu\mu}}$
    \item the norm of a vector is invariant under parallel transport.
\end{enumerate}
\subsubsection{Affine Connections}
\begin{definition}[Affine connection]
    An \textbf{affine connection} $\nabla$ is a map $\nabla: \chi(M)\times\chi(M)\to\chi(M)$, or $(X,Y)\mapsto\nabla_X Y$ which satisfies the following conditions:
    \begin{align}
        \nabla_X(Y+Z)   & =\nabla_X Y+\nabla_X Z  \\
        \nabla_{(X+Y)}Z & =\nabla_{X}Z+\nabla_Y Z \\
        \nabla_{(fX)}Y  & =f\nabla_{X}Y\notag     \\
        \nabla_X (fY)   & =X[f]Y+f\nabla_X Y
    \end{align}
    where $f\in\mathcal{F}(M)$ and $X,Y,Z\in\chi(M)$.
\end{definition}

Take a chart $(U,\phi)$ with the coordinate $x=\phi(p)$ on $M$, and define $m^3$ functions $\Gamma_{\nu\mu}^{\lambda}$ called the \textbf{connection coefficients} by
\begin{align}
    \nabla_\nu e_\mu\equiv\nabla_{e_\nu}e_\mu=e_\lambda\tensor{\Gamma}{^{\lambda}_{\nu\mu}}
\end{align}
where $\{e_\mu\}=\{\partial/\partial x^\mu\}$ is the coordinate basis in $T_p M$.

Let $V=V^\mu e_\mu$ and $W=W^\nu e_\nu$ be elements of $T_p(M)$.
Then
\begin{align}
    \nabla_V W=V^\mu\left(\frac{\partial W^\lambda}{\partial x^\mu}+W^\nu \tensor{\Gamma}{^\lambda_{\mu\nu}}\right)e_\lambda.
\end{align}
We can see that the components of $\nabla_VW$ do not depend on the derivative of components of $V$ namely $\partial V^\lambda/\partial x^\mu$, so $\nabla_V W$ can be seen to be a generalization of directional derivative to tensors.
\subsubsection{Parallel Transport and Geodesics}

\subsection{Curvature and Torsion}
\subsubsection{Definitions}
\begin{definition}[Torsion tensor]
    The \textbf{torsion tensor} is a map $T:\chi(M)\otimes\chi(M)\to\chi(M)$ such that
    \begin{align}
        T(X,Y)\equiv\nabla_X Y-\nabla_Y X-\comm{X}{Y}
    \end{align}
\end{definition}
\begin{definition}[Riemann curvature tensor (Riemann tensor)]
    The \textbf{Riemann curvature tensor} (or \textbf{Riemann tensor}) is a map $R:\chi(M)\otimes\chi(M)\otimes\chi(M)\to\chi(M)$ such that
    \begin{align}
        R(X,Y,Z)\equiv\nabla_X\nabla_Y Z-\nabla_Y\nabla_X Z-\nabla_{\comm{X}{Y}}Z.
    \end{align}
\end{definition}
\subsubsection{Geometrical Meaning of The Riemann Tensor and The Torsion Tensor}
\subsubsection{The Ricci Tensor and The Scalar Curvature}
\begin{definition}[Ricci tensor]
    The \textbf{Ricci tensor} $Ric$ is a type $(0,2)$ tensor defined by 
    \begin{align}
        Ric(X,Y)\equiv\expval{\dd{x^\mu},R(e_\mu,Y)X}
    \end{align} 
    whose component is 
    \begin{align}
        Ric_{\mu\nu}=Ric(e_\mu,e_\nu)=\tensor{R}{^\lambda_\mu_\lambda_\nu}.
    \end{align}
\end{definition}

\begin{definition}[Scalar curvature]
    The \textbf{scalar curvature} $\mathcal{R}$ is obtained by further contracting indices,
    \begin{align}
        \mathcal{R}\equiv g^{\mu\nu}Ric(e_\mu,e_\nu)=g^{\mu\nu}Ric_{\mu\nu}
    \end{align}
\end{definition}
\subsection{Levi-Civita Connections}
\subsubsection{The Fundamental Theorem}
\begin{theorem}[The fundamental theorem of (pseudo-)Riemannian geometry]
    On a (pseudo-)Riemannian manifold $(M,g)$, there exists a unique symmetric connection which is compatible with the metric $g$. 
    This connection is called the \textbf{Levi-Civita connection}.
\end{theorem}
\begin{proof}
    to be added
\end{proof}
\subsubsection{The Levi-Civita Connection in The Classical Geometry of Surfaces}
\subsubsection{Geodesics}
\subsubsection{The Normal Coordinate System}
\subsubsection{Riemann Curvature Tensor with Levi-Civita Connection}
\begin{theorem}[Bianchi identities]
    Let $R$ be the Riemann tensor defined with respect to the Levi-Civita connection. 
    Then $R$ satisfies the following identities:
    \begin{enumerate}
        \item The \textbf{first Bianchi identity}
        \begin{align}
            R(X,Y)Z+R(Z,X)Y+R(Y,Z)X=0.
        \end{align}
        \item The \textbf{second Bianchi identity}
        \begin{align}
            (\nabla_X R)(Y,Z)V+(\nabla_Z R)(X,Y)V+(\nabla_Y R)(Z,X)V=0.
        \end{align}
    \end{enumerate}
\end{theorem}
\begin{proof}
    to be added
\end{proof}
\subsection{Holonomy}
\begin{definition}[Holonomy group]
    Let $p$ be a point in $(M,g)$ and consider the set of closed loops at $p$, $\{c(t)|0\leq t\leq1,c(0)=c(1)=p\}$.
    Take a vector $X\in T_p M$ and parallel transport $X$ along a curve $c(t)$.
    After a trip along $c(t)$, we end up with a new vector $X_c\in T_p M$.
    Thus, the loop $c(t)$ and the connection $\nabla$ induce a linear transformation
    \begin{align}
        P_c:T_p M\to T_p M.
    \end{align}
    The set of these transformations is denoted by $H(p)$ and is called the \textbf{holonomy group} at $p$.
\end{definition}
\subsection{Isometries and Conformal Transformations}
\subsection{Killing Vector Fields and Conformal Killing Vector Fields}
\subsection{Non-coordinate Bases}
\subsection{Differential Forms and Hodge Theory}
\subsection{Aspects of General Relativity}
\subsubsection{Introduction to General Relativity}
Einstein proposed the following principles to construct the general theory of relativity
\begin{enumerate}
    \item \textbf{Principle of General Relativity}: All laws in physics take the same forms in any coordinate system.
    \item \textbf{Principle of Equivalence}: There exists a coordinate system in which the effect of a gravitational field vanishes locally.
\end{enumerate}

Any theory of gravity must reduce to Newton's theory of gravity in the weak-field limit. 
In Newton's theory, the gravitational potential $\Phi$ satisfies the Poisson equation
\begin{align}\label{nequation}
    \Delta_\phi=4\pi G\rho,
\end{align}
where $\rho$ is the mass density and $\Delta$ is the Laplace operator.

In general relativity, the gravitational potential is replaced by the components of the metric tensor.
Then, instead of the LHS of \cref{nequation}, we have the \textbf{Einstein tensor} defined by
\begin{align}
    G_{\mu\nu}\equiv Ric_{\mu\nu}-\frac{1}{2}g_{\mu\nu}\mathcal{R}
\end{align}
Similarly, the mass density is replaced by a more general object called the \textbf{energy-momentum tensor} $T_{\mu\nu}$.
\lec{Einstein equation}{}And the \textbf{Einstein equation} is \sidenote{The constant $8\pi G$ is chosen so that \cref{einsteinequation} reproduces the Newtonian result in the weak-field limit.}
\begin{align}\label{einsteinequation}
    G_{\mu\nu}=8\pi GT_{\mu\nu}.
\end{align}
\begin{remark}
    The tensor $T_{\mu\nu}$ is obtained from the matter action by the variational principle.
    From Noether's theorem, $T_{\mu\nu}$ must satisfy a conservation equation of the form $\nabla_{\mu}T^{\mu\nu}=0$.
\end{remark}

\subsubsection{Einstein-Hilbert Action}
We define the \textbf{Einstein-Hilbert action} by 
\begin{align}
    S_{\text{EH}}\equiv\frac{1}{16\pi G}\int\mathcal{R}\sqrt{-g}\dd[4]{x}.
\end{align}


\subsection{Bosonic String Theory}




\newpage
\bibliographystyle{jhep}
\bibliography{ref}
\end{document}