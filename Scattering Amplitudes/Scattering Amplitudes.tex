\documentclass[10pt]{article}
\usepackage{NotesTeXV3}
\usepackage{orcidlink}
\usepackage{tensor}
\usepackage{cleveref}
\usepackage{slashed}
\usepackage[export]{adjustbox}
\usepackage{silence}

\WarningFilter{latex}{Marginpar on page}

\hbadness=99999

\let\oldsection\section
\def\section{%
  \setcounter{sidenote}{1}%
  \oldsection
}

\begin{document}
\title{Scattering Amplitudes}
\author{Shangjie Zhou\orcidlink{0000-0001-9576-5011}}
\affiliation{School of Physics and Technology, Wuhan University}
\emailAdd{sjzhou@whu.edu.cn}
\abstract{\textit{Last updated on: \today}\\\url{https://github.com/spaceofzsj/notes}}

\maketitle
\section{Introduction}
Scattering experiments are crucial for our understanding of the building blocks of nature.
The standard model of particle physics was developed from scattering experiments, including the discovery of the weak force bosons $W^{\pm}$ and $Z^0$, quarks and gluons, and most recently the Higgs boson.

The key observable measured in particle scattering experiments is the scattering cross-section $\sigma$.
It encodes the likelihood of a given process to take place as a function of the energy and momentum of the particles involved.
A more refined version of this quantity is the differential cross-section $\dd{\sigma}/\dd{\Omega}$: it describes the dependence of the cross-section on the angles of the scattered particles.

Interpretation of data from scattering experiments relies heavily on theoretical predictions of scattering cross-sections.
These are calculated in relativistic quantum field theory (QFT), which is the mathematical language for describing elementary particles and their interactions.
Relativistic QFT combines special relativity with quantum physics and is a hugely successful and experimentally well-tested framework for describing elementary particles and the fundamental forces of nature.
In quantum mechanics, the probability distribution $\abs{\psi}^2=\psi^* \psi$ for a particle is given by the norm-squared of its complex-valued wavefunction $\psi$.
Analogously, in quantum field theory, the differential cross-section is proportional to the norm-squared of the scattering amplitude $A$, $\dd{\sigma}/\dd{\Omega}\propto\abs{A}^2$.
The amplitudes A are well-defined physical observables.

Each scattering process is characterized by the types of particles involved in the initial and final states as well as the relativistic momentum $\vec{p}$ and energy $E$ of each particle.
This input is the external data for an amplitude: a scattering amplitude $A_n$ involving a total of $n$ initial and final state particles takes the list $\left\{E_i, \vec{p}_i; \text{type}_i\right\}, i=1,2,\dots,n$ of external data and returns a complex number:
\begin{align}
    n\text{-particle amplitude}\ A_n:\quad \left\{E_i, \vec{p}_i; \text{type}_i\right\}\rightarrow A_n\left(\left\{E_i, \vec{p}_i; \text{type}_i\right\}\right)\in\mathbb{C}.
\end{align}
“Particle type” involves more than saying which particles scatter: it also includes a specification of the appropriate quantum numbers of the initial and final states, for example the polarization of a photon or the spin-state of a fermion.

This note follows \cite{Elvang:2015rqa}.

There are also many books or reviews on this topic such as \cite{Adamo:2022dcm}\cite{Cheung:2017pzi}\cite{Henn:2014yza}\cite{Taylor:2017sph}.

\clearpage
\part{Trees}
\section{Spinor Helicity Formalism}
The spinor helicity formalism is a highly convenient and powerful notational tool for amplitudes of massless particles in 4-dimensional theories.
There are many papers which are regarded as earliest developments of spinor helicity formalism.

The Lagrangian for a free massive 4-component Dirac field $\psi$ is
\begin{align}
    \mathcal{L}=i\bar{\psi}(\slashed{\partial}-m)\psi.
\end{align}
The equation of motion gives Dirac equation
\begin{align}\label{eq:dirac}
    (-i\slashed{\partial}+m)\psi=0.
\end{align}
Multiplying the Dirac equation by $(i\slashed{\partial}+m)$ gives Klein-Gordon equation, $(-\partial^2+m^2)\psi=0$. It can be solved by a plane-wave expansion ansatz
\begin{align}\label{s}
    \psi(x)= u(p)e^{ip\cdot x}+v(p)e^{-ip\cdot x}.
\end{align}
Put \cref{s} in \cref{eq:dirac} and we have
\begin{align}\label{diracmomentum}
    (\slashed{p}+m)u(p)=0\qq{and}(-\slashed{p}+m)v(p)=0.
\end{align}
\Cref{diracmomentum} are called Dirac equation in momentum space.
Each of the equations in \cref{diracmomentum} has two solutions which we will label by a subscript $s=\pm$.
We can now write the general free field expansion of $\psi$ as
\begin{align}
    \psi(x)=\sum_{s=\pm}\int\widetilde{\dd p}\left[b_s(p) u_s(p)e^{ip\cdot x}+d^\dagger_s(p)v_s(p)e^{-ip\cdot x}\right],
\end{align}
where $\widetilde{\dd p}=\frac{\dd[3]{p}}{(2\pi)^3 2 E_p}$ is the 3d Lorentz-invariant momentum measure.

In canonical quantization, $(b^\dagger_\pm(p),d^\dagger_\pm(p))$ and $(b_\pm(p),d_\pm(p))$ become fermionic creation and annihilation operators.
The next step in canonical quantization is to define the vacuum $\ket{0}$ such that $b_{\pm}(0)\ket{0}=0$ and $d_{\pm}(p)\ket{0}=0$.
One-particle states are then defined as $\ket{p;\pm}\equiv d^\dagger_{\pm}(p)\ket{0}$, etc.

We write the 4-momentum $p^\mu$ of an on-shell particle as a matrix in spinor indices via
\begin{align}
    \slashed{p}=\mqty(0 & p_{a\dot{b}} \\p^{\dot{a}b}&0),
\end{align}
with
\begin{align}
    p_{a\dot{b}}\equiv p_\mu(\sigma^\mu)_{a\dot{b}}=\mqty(-p^0+p^3 & p^1-ip^2 \\p^1+ip^2&-p^0-p^3),
\end{align}
and similarly $p^{\dot{a}b}\equiv p_\mu\left(\bar{\sigma}^\mu\right)^{\dot{a}b}$,
where $\sigma^{\mu}\equiv(1,\sigma^i)$ and $\bar{\sigma}^\mu\equiv(1,-\sigma^i)$.
Then the on-shell condition now becomes
\begin{align}\label{eq:detp}
    \det p=-p^\mu p_\mu=m^2.
\end{align}
\begin{remark}
    Under a Lorentz transformation $\Lambda$, we have $p^\mu\to \tensor{\Lambda}{^\mu_\nu}p^\nu$. Notice that in chiral representation
    \begin{align}
        \tensor{\Lambda}{^\mu_\nu}\gamma^\nu=S^{-1}_{\text{Dirac}}(\Lambda)\gamma^\mu S_{\text{Dirac}}(\Lambda)
    \end{align}
    and recall that
    \begin{align}
        S_{\text{Dirac}}(\Lambda)=\mqty(\Lambda_{(\frac{1}{2},0)} & 0 \\0&\Lambda_{(0,\frac{1}{2})})
    \end{align}
    $\slashed{p}$ transforms in the following way
    \begin{align}
        \slashed{p}=\mqty(0 & p_{a\dot{b}}                                                                                                                                      \\p^{\dot{a}b}&0)\to p_\beta \tensor{\Lambda}{^\beta_\mu}\gamma^\mu&=p_\beta S^{-1}_{\text{Dirac}}(\Lambda)\gamma^\beta S_{\text{Dirac}}(\Lambda)\notag\\
                            & = S^{-1}_{\text{Dirac}}(\Lambda)\slashed{p} S_{\text{Dirac}}(\Lambda)\notag                                                                       \\
                            & =\mqty(0                                                                    & \Lambda^{-1}_{(\frac{1}{2},0)}p_{a\dot{b}}\Lambda_{(0,\frac{1}{2})} \\\Lambda^{-1}_{(0,\frac{1}{2})}p^{\dot{a}b}\Lambda_{(\frac{1}{2},0)}&0).
    \end{align}
    So it is clear that the indices on $p$ are indeed spinor indices.
\end{remark}


When $m=0$, the Dirac equation in momentum space reads
\begin{align}\label{massless}
    \slashed{p}v_\pm(p)=0\qq{and}\bar{u}_\pm(p)\slashed{p}=0.
\end{align}
\textit{Crossing symmetry} exchanges (incoming $\leftrightarrow$ outcoming), (fermions $\leftrightarrow$ anti-fermions) and flips the sign of the helicity, so in the massless case the wavefunctions are related as $u_\pm=v_\mp$ and $\bar{v}_\pm=\bar{u}_\mp$.

Moreover, \cref{eq:detp} now becomes
\begin{align}
    \det p=0,
\end{align}
which implies that $\text{rank of}\ p^{\dot{a}b}<2$ and we have $p^{\dot{a}b}=\tilde{\lambda}^{\dot{a}}\lambda^b$.
The indices of $\tilde{\lambda}^{\dot{a}}$ and $\lambda^a$ are defined to be raised and lowered with the invariant Levi-Civita tensor\sidenote{$\lambda_a$ and $\tilde{\lambda}^{\dot{a}}$ are column vectors. $\tilde{\lambda}_{\dot{a}}$ and $\lambda^a$ are row vectors}:
\begin{align}
    \lambda_\alpha\equiv\epsilon_{\alpha\beta}\lambda^\beta\qq{and}\tilde{\lambda}_{\dot{\alpha}}\equiv\epsilon_{\dot{\alpha}\dot{\beta}}\tilde{\lambda}^{\dot{\beta}}.
\end{align}
It can be verified that \cref{massless} can be solved by
\begin{align}
    v_+(p)=\mqty(\lambda_a \\0)\equiv\mqty(|p]_a\\0)\qq{and}v_-(p)=\mqty(0\\\tilde{\lambda}^{\dot{a}})\equiv\mqty(0\\\ket{p}^{\dot{a}}),
\end{align}
and
\begin{align}
    \bar{u}_-(p)=\mqty(0 & \tilde{\lambda}_{\dot{a}})\equiv\mqty(0 & \bra{p}_{\dot{a}})\qq{and}\bar{u}_+(p)=\mqty(\lambda^a & 0)\equiv\mqty([p|^a & 0).
\end{align}
The angle and square spinors are the core of what we call the \textbf{spinor helicity formalism}.

\paragraph{Reality conditions.}
From the crossing symmetry $\bar{u}_\mp=\bar{v}_\pm$, if the momentum is real\sidenote{The crossing symmetry applies to real particles which have real momentum.}, we have
\begin{align}
    [p|^a=\left(\ket{p}^{\dot{a}}\right)^*\qq{and}\bra{p}_{\dot{a}}=\left(|p]_a\right)^*.
\end{align}

\paragraph{Spinor completeness relation.}
The spin-sum completeness relation with $m=0$ which reads $u_-\bar{u}_-+u_+\bar{u}_+=-\slashed{p}$ now becomes
\begin{align}
    p_{a\dot{b}}=-|p]_a\bra{p}_{\dot{b}}\qq{and}p^{\dot{a}b}=-\ket{p}^{\dot{a}}[p|^b.
\end{align}

\lec{Feynman Rules}{Fermions}The Feynman rules of fermions can be conveniently expressed with spinor variables
\begin{itemize}
    \item Outgoing fermion with $h=+1/2$: $u_+(p)\leftrightarrow\mqty([p|^a&0)$;
    \item Outgoing fermion with $h=-1/2$: $u_-(p)\leftrightarrow\mqty(0&\bra{p}_{\dot{a}})$;
    \item Ingoing fermion with $h=+1/2$: $v_+(p)=\leftrightarrow\mqty(|p]_a\\0)$;
    \item Ingoing fermion with $h=-1/2$: $v_-(p)=\leftrightarrow\mqty(0\\\ket{p}^{\dot{a}})$.
\end{itemize}

\clearpage

\section{Yang-Mills Theory and Color Decomposition}
Gluons are described by the Yang-Mills Lagrangian
\begin{align}
    \mathcal{L}=-\frac{1}{4}\Tr F_{\mu\nu}F^{\mu\nu},
\end{align}
with $F_{\mu\nu}=\partial_\mu A_\nu-\partial_{\mu}A_\mu-\frac{ig}{\sqrt{2}}\comm{A_\mu}{A_\nu}$ and $A_\mu=A^a_{\mu}T^a$.

The gauge group $G=\text{SU}(3)$ for QCD, but we will keep the number of colors $N_c$ general and take $G=\text{SU}(N_c)$.
The gluon fields are in the adjoint representation, so the color-indices run over $a,b,\dots=1,2,\dots,N_c^2-1$.
The generators $T^a$ are normalized such that $\Tr(T^a T^b)=\delta^{ab}$ and $\comm{T^a}{T^b}=i\tilde{f}^{abc}T^{c}$.

\lec{Feynman Rules}{Yang-Mills Theory}In \textit{Gervais-Neveu gauge} for which the gauge-fixing term is $\mathcal{L}_{gf}=-\frac{1}{2}\Tr\left(\tensor{H}{_\mu^\mu}\right)^2$ with $H_{\mu\nu}=\partial_\mu A_\nu-\frac{ig}{\sqrt{2}}A_\mu A_\nu$,
the Lagrangian takes the form\sidenote{We ignore the ghosts here because only the tree-level amplitudes are considered.}
\begin{align}
    \mathcal{L}=\Tr(-\frac{1}{2}\partial_\mu A_\nu \partial^\mu A^\nu-i\sqrt{2}g\partial^\mu A^\nu A_\nu A_\mu+\frac{g^2}{4}A^\mu A^\nu A_\nu A_\mu).
\end{align}
The Feynman rules then give a gluon propagator $\delta^{ab}\frac{\eta_{\mu\nu}}{p^2}$.
The 3- and 4-vertices involve group factors $\tilde{f}^{abc}$ and $\tilde{f}^{abe}\tilde{f}^{ecd}$.

The amplitudes constructed from these rules can be organized into different group theory structures each dressed with a kinematic factor.
For example, the \textit{color factors} of the \textit{s-}, \textit{t-}, and \textit{u-}channel diagram of the 4-gluon tree amplitude are
\begin{align}
    c_s\equiv\tilde{f}^{a_1 a_2 b}\tilde{f}^{b a_3 a_4}\qc c_t\equiv\tilde{f}^{a_1 a_3 b}\tilde{f}^{b a_4 a_2}\qc c_u\equiv\tilde{f}^{a_1 a_4 b}\tilde{f}^{b a_2 a_3},
\end{align}
and the 4-point contact term generically gives a sum of contributions with $c_s,c_t$ and $c_u$ color factors.
The \textit{Jacobi identity} relates the three color factors:
\begin{align}
    c_s+c_t+c_u=0.
\end{align}
So there are only two independent color-structures for the tree-level 4-gluon amplitude.

With the normalzation condition, We can write the color factors in terms of traces of the generators $T^a$:
\begin{align}
    i\tilde{f}^{abc}=\Tr(T^a T^b T^c)-\Tr(T^b T^a T^c).
\end{align}
\begin{proof}
    \begin{align}
        \comm{T^a}{T^b}                   & =i\tilde{f}^{ab\alpha}T^\alpha \notag         \\
        \comm{T^a}{T^b}T^c                & =i\tilde{f}^{ab\alpha}T^\alpha T^c\notag      \\
        \Tr(\comm{T^a}{T^b}T^c)           & =i\tilde{f}^{ab\alpha}\Tr(T^\alpha T^c)\notag \\
        \Tr(T^a T^b T^c)-\Tr(T^b T^a T^c) & =i\tilde{f}^{abc}\notag
    \end{align}
\end{proof}

Products of traces can be simplified using the completeness relation
\begin{align}
    \tensor{\left(T^a\right)}{_i^j}\tensor{\left(T^a\right)}{_k^l}=\tensor{\delta}{_i^l}\tensor{\delta}{_k^j}-\frac{1}{N_c}\tensor{\delta}{_i^j}\tensor{\delta}{_k^l}.
\end{align}
\begin{proof}
    123
\end{proof}
\begin{example}[4-gluon case]\leavevmode
    For the 4-gluon $s$-channel diagram we have
    \begin{align}
        \tilde{f}^{a_1 a_2 b}\tilde{f}^{b a_3 a_4}= & \Tr(T^{a_1}T^{a_2}T^{a_3}T^{a_4})-\Tr(T^{a_1}T^{a_2}T^{a_4}T^{a_3})\notag \\
                                                    & +\Tr(T^{a_1}T^{a_3}T^{a_4}T^{a_2})-\Tr(T^{a_1}T^{a_4}T^{a_3}T^{a_2}).
    \end{align}
    Similarly, the three other diagrams contributing to the 4-gluon amplitude can also be written in terms of single-trace group theory factors.
    So that means that we can write the 4-gluon tree amplitude as
    \begin{align}
        A^{\text{full,tree}}_4=g^2\left(A_4[1234]\Tr(T^{a_1}T^{a_2}T^{a_3}T^{a_4})+\text{perms of }(234)\right),
    \end{align}
    where the \textit{partial amplitudes} $A_4[1234]$, $A_4[1243]$ etc. are called \textbf{color-ordered amplitudes}\sidenotemark.
\end{example}
\sidenotetext{We use the squared parenthesis in $A_4[1234]$ to distinguish the color-ordered amplitude from the non-color-ordered amplitude $A_4(1234)$.}
The color-structure generalizes to any $n$-point tree-level amplitude involving any particles that transform in the adjoint of the gauge group: we write
\begin{align}\label{fulltree}
    A^{\text{full,tree}}_n=g^{n-2}\sum_{\text{perms}\ \sigma}A_n\left[1\sigma(2\dots n)\right]\Tr(T^{a_1}T^{\sigma(a_2}\cdots T^{a_n)}),
\end{align}
where the sum is over the (overcomplete) trace-basis of $(n-1)!$ elements that takes into account the cyclic nature of the traces\sidenote{For loop-amplitudes, one also needs to consider multi-trace structures in addition to the simple single-trace.}.
\begin{remark}
    The form of the color-ordered amplitudes do not depend on the type of the gauge group.

    \Cref{fulltree} is a general result for all gauge groups.
\end{remark}

The Feynman vertex rules for calculating the \textit{color-ordered} amplitudes\sidenote{Using this set of rules, we can write down the color-ordered amplitudes directly by putting the particles on external lines of the Feynman diagram in the corresponding color order and summing over all possible diagrams.} are:
\begin{align}
     & V^{\mu_1 \mu_2 \mu_3}(p_1,p_2,p_3)=\includegraphics[valign=c,width=1in]{fig/yang-mills/3point/3point.pdf}=-\sqrt{2}g\left(\eta^{\mu_1 \mu_2}p_1^{\mu_3}+\eta^{\mu_2 \mu_3}p_2^{\mu_1}+\eta^{\mu_3 \mu_1}p_3^{\mu_2}\right) \\
     & V^{\mu_1 \mu_2 \mu_3 \mu_4}(p_1,p_2,p_3,p_4)=\includegraphics[valign=c,width=1in]{fig/yang-mills/4point/4point.pdf}=g^2 \eta^{\mu_1 \mu_3}\eta^{\mu_2 \mu_4}.
\end{align}
\begin{property}[Color-ordered amplitudes]\leavevmode
              \begin{itemize}
                  \item \textit{Gauge invarance}: the color-ordered amplitudes are gauge invariant.

                        The single traces in \cref{fulltree} are linear independent\cite{Taylor:2017sph}.

                        The "linear independence" here means that there exists no non-trivial set of numbers $\{\alpha_i\}$ such that
                        \begin{align}
                            \sum_{\sigma_i}\sum_{j=1}^{N_c^2-1} \alpha_j \Tr(T^{a_1}T^{\sigma_j(a_2}\cdots T^{a_n)})=0
                        \end{align}
                        holds for generators $\{T^i\}$ of any gauge group,
                        where the sum runs over all possible permutations $\sigma_i$.
                  \item \textit{Cyclic}: $A_n[12\dots n]=A_n[2\dots n1]$, etc.
                  \item \textit{Reflection}: $A_n[12\dots n]=(-1)^n A_n[n\dots 21]$.

                        See \cite{Taylor:2017sph} for a detailed discussion.
                  \item The U(1) \textit{decoupling identity} (also known as \textit{Kleiss-Kuijf relation} or \textit{photon decoupling identity}):
                        \begin{align}
                            A_n[123\dots n]+A_n[213\dots n]+A_n[231\dots n]+\cdots+A_n[23\dots1n]=0.
                        \end{align}

                        Since \cref{fulltree} holds for any gauge group, if one of gauge bosons in the tree amplitude describing the scattering of $N$
                        U($N_c$) gauge bosons is associated to the $T^0=I_{N_c}$, the U(1) subgroup generator of U($N_c$),
                        the amplitude vanishes because $\tilde{f}_{0ab}=0$ for any color $a$ and $b$.
                        In other words, this abelian gauge boson does not couple to gluons.
                        Let's assume that it is the $N^{\text{th}}$ gauge boson.
                        Then according to \cref{fulltree}, we have
                        \begin{align}
                            A^{\text{full,tree}}_{N} & =\begin{aligned}[t]
                                g^{N-2}\sum_{\text{perms}\ \sigma} & A_N[12_\sigma\dots(N-1)_{\sigma}N]\Tr(T^1 T^{2_\sigma} \dots T^{(N-1)_\sigma} T^N)  \notag                \\
                                +                                  & A_{N}[12_\sigma\dots N(N-1)_\sigma]\Tr(T^1 T^{2_\sigma} \dots T^{N} T^{(N-1)_\sigma})             \notag  \\
                                                                   & \dots                                                                                              \notag \\
                                +                                  & A_{N}[1N2_\sigma\dots (N-1)_\sigma]\Tr(T^1 T^N T^{2_\sigma}\dots  T^{(N-1)_\sigma})  \notag
                            \end{aligned} \\
                                                     & \begin{aligned}[t]
                                = g^{N-2}\sum_{\text{perms}\ \sigma} & \left(A_N[12_\sigma\dots(N-1)_{\sigma}N]\Tr(T^1 T^{2_\sigma} \dots T^{(N-1)_\sigma} )\right.       \notag \\
                                +                                    & A_{N}[12_\sigma\dots N(N-1)_\sigma]\Tr(T^1 T^{2_\sigma} \dots  T^{(N-1)_\sigma})                  \notag  \\
                                                                     & \dots                                                                                              \notag \\
                                +                                    & \left.A_{N}[1N2_\sigma\dots (N-1)_\sigma]\Tr(T^1 T^{2_\sigma}\dots  T^{(N-1)_\sigma})\right)      \notag  \\
                            \end{aligned}  \\
                                                     & \begin{aligned}[t]
                                = g^{N-2}\sum_{\text{perms}\ \sigma}  \Tr(T^1 T^{2_\sigma}\dots  T^{(N-1)_\sigma}) & \left(A_N[12_\sigma\dots(N-1)_{\sigma}N]\right.       \notag       \\
                                +                                                                                  & A_{N}[12_\sigma\dots N(N-1)_\sigma]                         \notag \\
                                                                                                                   & \dots                                          \notag              \\
                                +                                                                                  & \left.A_{N}[1N2_\sigma\dots (N-1)_\sigma]\right)        \notag     \\
                            \end{aligned}  \\
                                                     & =0.\notag
                        \end{align}
                        Since the traces are linear independent, we have
                        \begin{align}
                             & A_N[12\dots(N-1)N]+A_{N}[12\dots N(N-1)]+\dots+A_{N}[1N2\dots (N-1)]\notag \\
                             & =0.\notag
                        \end{align}
              \end{itemize}
\end{property}

The trace-basis in \cite{Taylor:2017sph} is overcomplete and that implies that there are further linear relations among the \textit{partial tree-level} amplitudes:
these are called the \textbf{Kleiss-Kuijf relations (KK relations)}\cite{Kleiss:1988ne}\cite{DelDuca:1999rs} and they can be written\cite{Bern:2008qj}
\begin{align}\label{kk}
    A_{n}[1,\{\alpha\},n,\{\beta\}]=(-1)^{\abs{\beta}}\sum_{\sigma\in\text{OP}(\{\alpha\},\{\beta^T\})}A_n[1,\sigma,n],
\end{align}
where $\{\beta^T\}$ denotes the reverse ordering of the labels $\{\beta\}$ and the sum is over ordered permutations "OP",
namely permutations of the labels in the joined set $\{\alpha\}\cup\{\beta^T\}$ such that the ordering within $\{\alpha\}$ and $\{\beta^T\}$ is preserved.
$\abs{\beta}$ is the number of elements in $\{\beta\}$.
\begin{example}[5-point color-ordered amplitude]
    Take the LHS of \cref{kk} to be $A_5[1,\{2\},5,\{3,4\}]$, we have $\{\alpha\}\cup\{\beta^T\}=\{2\}\cup\{4,3\}$,
    so the sum over ordered permutations is over $\sigma=\{243\},\{423\},\{432\}$.
    Thus the Kleiss-Kuijf relation reads
    \begin{align}
        A_5[12534]=A_5[12435]+A_5[14235]+A_5[14325].
    \end{align}
\end{example}
\begin{remark}
    The Kleiss-Kuijf relations combine with the cyclic, reflection, and U(1) decoupling identities to reduce the number independent $n$-gluon tree amplitudes to $(n-2)!$.

    There are further linear relationships, called the (fundamental) \textbf{BCJ relations}\cite{Bern:2008qj}
    that reduce the number of independent $n$-gluon color-ordered tree amplitudes to $(n-3)!$.
\end{remark}


\clearpage

\section{Little Group Scaling}

\clearpage
\section{On-shell Recursion Relations at Tree-level}

\subsection{Complex Shifts and Cauchy's Theorem}

\subsection{BCFW Recursion Relations}

\clearpage
\section{Supersymmetry}

\clearpage
\section{Symmetries of \texorpdfstring{{}$\mathcal{N}=4$}{} SYM}


\clearpage



\part{Loops}

\section{Loop Amplitudes and Generalized Unitarity}
In general, an $L$-loop amplitude can be written \textit{schematically} as
\begin{align}
    \mathcal{A}^{L-\text{loop}}_{n}=\sum_j\int\left(\prod_{l=1}^{L}\frac{\dd[D]{\ell_l}}{(2\pi)^D}\right)\frac{1}{S_j}\frac{n_j c_j}{\prod_{\alpha_j}p^{2}_{\alpha_j}},
\end{align}
where
\begin{itemize}
    \item $j$: runs over all possible $L$-loop Feynman diagrams $j$
    \item $\ell_l$: the $L$ loop-momenta
    \item $S_j$: symmetry factor associated with the diagram
    \item $n_j$: kinematic numerator factors, which are polynomials of Lorentz-invariant contractions of external- or loop-momenta and polarization vectors (or other external wavefunctions).
    \item $c_j$: couplings and gauge group factors.
\end{itemize}

We will be only concerned with the \textit{planar limit} of Yang-Mills amplitudes.
In this context, the planar limit means that the amplitudes have only one trace-structure, just as the tree amplitudes.
They are called \textit{planar amplitudes}.

\clearpage
\section{Unitarity Method}

\clearpage

\part{Topics}
\section{Supergravity Amplitudes}
\subsection{Perturbative Gravity}
The Einstein equation $G_{\mu\nu}=8\pi T_{\mu\nu}$, is the classical equation of motion that follows from the variational principle applied to the \textit{Einstein-Hilbert action}
\begin{align}\label{ehaction}
    S_{\text{EH}}=\frac{1}{2\kappa^2}\int\dd[D]{x}\sqrt{-g}R+S_{\text{matter}},
\end{align}
where $R$ is the Ricci scalar and $2\kappa^2=16\pi G_N$.
We have written the action in $D$ spacetime dimensions with a $D$-dimensional Newton's constant $G_N$.
The metric $g_{\mu\nu}$ is a field in the field theory described by \cref{ehaction}.
The varition $\delta g_{\mu\nu}$ of $\sqrt{-g}R$ gives the Einstein tensor part, $G_{\mu\nu}=R_{\mu\nu}-\frac{1}{2}g_{\mu\nu}R$, of Einstein's equation,
while the metric variation of the "matter" action $S_{\text{matter}}$ gives the stress-tensor part, $T_{\mu\nu}=\frac{2}{\sqrt{-g}}\fdv{S_{\text{matter}}}{g^{\mu\nu}}$.
In the following, we use the term \textbf{pure gravity} to describe the field theory \cref{ehaction} without matter fields, $S_{\text{matter}}=0$.

We will consider the quantization of the metric field $g_{\mu\nu}$ in the flat spacetime.
To be specific, we expand the gravitational field around flat spacetime $g_{\mu\nu}=\eta_{\mu\nu}+\kappa h_{\mu\nu}$ and regard the fluctuating field $h_{\mu\nu}$ as the \textbf{graviton field}.

Suppressing the increasingly intricate index-structure, we write these terms schematically as $h^{n-1}\partial^2 h$ for $n=2,3,4\dots$, so that the action \cref{ehaction} becomes
\begin{align}\label{ehexpand}
    S_{\text{EH}}=\int\dd[D]{x}\left[h\partial^2h+\kappa h^2 \partial^2 h+\kappa^2 h^3\partial^2 h+\kappa^3 h^4\partial^2 h+\dots\right].
\end{align}
There are infinitely many terms and no mass terms in \cref{ehexpand}.
So the particles associated with quantization of the gravitational field $h_{\mu\nu}$ are massless and have spin-2, which are called \textbf{gravitons}.

It typical to use the \textit{de Donder gauge}, $\partial^\mu h_{\mu\nu}=\frac{1}{2}\partial_\nu \tensor{h}{_\mu^\mu}$, to extract Feynman rules from \cref{ehexpand}, which brings the quadratic terms in the action to the form
\begin{align}
    h\partial^2 h\to-\frac{1}{2}h_{\mu\nu}\Box h^{\mu\nu}+\frac{1}{4}\tensor{h}{_\mu^\mu}\Box \tensor{h}{_\nu^\nu}.
\end{align}
\lec{Feynman rules}{graviton}The propagator resulting from these quadratic terms is
\begin{align}
    P_{\mu_1\nu_1,\mu_2\nu_2}=-\frac{i}{2}\left(\eta_{\mu_1\mu_2}\eta_{\nu_1\nu_2}+\eta_{\mu_1\nu2}\eta_{\nu_1\mu_2}-\frac{2}{D-2}\eta_{\mu_1\nu_1}\eta_{\mu_2\nu_2}\right)\frac{1}{k^2},
\end{align}
where $k$ is the momentum on the propagator.

External lines can be constructed as products of the spin-1 polarization vectors
\begin{align}
    e^{\mu\nu}_-(p_i)=\epsilon^\mu_-(p_i)\epsilon^\nu_+(p_i)\qq{and}e^{\mu\nu}_+(p_i)=\epsilon^\mu_+(p_i)\epsilon^\nu_-(p_i).
\end{align}
The infinite set of 2-derivative interaction terms $h^{n-1}\partial^2 h$ yields Feynman rules for $n$-graviton vertices for \textit{any} $n=3,4,5,\dots$.
For example, the de Donder gauge 3-vertex takes the form
\begin{align}
    V_3(p_1,p_2,p_3)=p^{\mu_3}_1p^{\nu_3}_2\eta^{\mu_1 \nu_2}\eta^{\mu_2\nu_1}+(\text{many other terms with various index-structures}).
\end{align}
Feynman rules of graviton are too complicated to show here.
For the detailed discussion, see \cite{DeWitt:1967yk}\cite{DeWitt:1967ub}\cite{DeWitt:1967uc}\cite{Veltman:1975vx}.

Because of the extremely complicated Feynman rules, it is even nearly impossible to calculate the graviton amplitudes to tree-level using Feynman diagrams.
In \cite{Kawai:1985xq}, the 4-point graviton tree amplitude was calculated brute force with Feynman diagrams where each of the four contributing Feynman diagrams fill about a page or so.
Nonetheless, the final result can be brought to a very simple form:  in 4D, it can be written in spinor helicity formalism as\sidenote{Note that we will be using $M_n$ to denote (super)gravity amplitudes to distinguish them from (super)Yang-Mills amplitudes $A_n$.}
\begin{align}
    M^{\text{tree}}_4\left(1^-2^-3^+4^+\right)=\frac{\langle12\rangle^7[12]}{\langle13\rangle\langle14\rangle\langle23\rangle\langle24\rangle\langle34\rangle^2}=\frac{\langle12\rangle^4[34]^4}{stu}.
\end{align}

\begin{intu}
    Little group scaling fixes the possible 3-graviton amplitudes and recursion then allows us to compute all other tree-level graviton processes.
    Loop-level amplitudes can be constructed with unitarity techniques. 
    Thus the infinite set of interaction terms in the Lagrangian is not needed from the point of view of the on-shell scattering amplitudes.
\end{intu}

In the $D=4$ case, dimensional analysis and little group scaling fix the 3-point graviton amplitudes to be
\begin{align}
    &M_3(1^-2^-3^+)=\frac{\langle12\rangle^6}{\langle23\rangle^2\langle31\rangle^2}=A_3[1^-2^-3^+]^2\\
    &M_3(1^+2^+3^-)=\frac{[12]^6}{[23]^2[31]^2}=A_3[1^+2^+3^-]^2.
\end{align}

Using BCFW recursion relations, relatively compact graviton amplitudes can be found for the MHV sector
\begin{align}
    &M_n^{\text{tree}}(1^-2^-3^+\dots n^+)\notag\\
    &=\sum_{P(3,4,\dots,n-1)}\frac{\langle12\rangle^8\prod_{l=3}^{n-1}\langle n|2+3+\cdots+(l-1)|l]}{\left(\prod_{i=1}^{n-2}\langle i,i+1\rangle\right)\langle 1,n-1\rangle\langle1n\rangle^2\langle2n\rangle^2\left(\prod_{l=3}^{n-1}\langle ln\rangle\right)}.
\end{align}
The sum is over all permutations of labels $(3,4,\dots,n-1)$.

Another form of the same MHV graviton tree amplitude expresses $M^{\text{tree}}_n$ in terms of color-ordered gluon amplitudes $A^{\text{tree}}_n$ of Yang-Mills theory
\begin{align}\label{mhvgraviton}
    M_n^{\text{tree}}(1^-2^-3^+\dots n^+)=\sum_{P(i_3,i_4,\dots,i_n)}s_{1i_n}\left(\prod^{n-1}_{k=4}\beta_k\right)A^{\text{tree}}_n\left[1^-2^- i_3^+ i_4^+\dots i_n^+\right]^2,
\end{align}
where $n\geq 4$ and 
\begin{align}
    \beta_k=-\frac{\langle i_k i_{k+1}\rangle}{\langle 2 i_{k+1}\rangle}\langle2|i_3+i_4+\cdots+i_{k-1}|i_k].
\end{align}
\Cref{mhvgraviton} shows that the amplitudes of gravitons and gauge theory are closely related.
The first such example is the \textbf{KLT relations}\cite{Kawai:1985xq}.
For $n=4$ and $5$ the KLT relations are
\begin{align}
    M^{\text{tree}}_4(1234)&=-s_{12}A^{\text{tree}}_4[1234]A^{\text{tree}}_4[1243]\label{kltexample1}\\
    M^{\text{tree}}_5(12345)&=s_{23}s_{45}A^{\text{tree}}_5[12345]A^{\text{tree}}_{5}[13254]+(3\leftrightarrow4).\label{kltexample2}
\end{align}
More KLT relations can be found in \cite{Bern:1998sv}.
\begin{remark}
    There is no specification of helicities of the external states in KLT relations \cref{kltexample1} and \cref{kltexample2} because the KLT relations are valid in $D$-dimensions.
    In 4d, the KLT relations work for any helicity assignments of the gravitons on the LHS;
    if the $i$th graviton has helicity $h_i=+2$, then the gluons labeled $i$ in the amplitudes on the RHS have helicity $h_i=+1$; similarly for negative helicity.
\end{remark}
We may then say that KLT in 4d uses
\begin{align}
    \text{graviton}^{\pm2}(p_i)=\text{gluon}^{\pm1}(p_i)\otimes\text{gluon}^{\pm1}(p_i).
\end{align}


\subsection{Supergravity}

\clearpage
\section{Color-kinematic Duality}
\subsection{The Color-structure of Yang-Mills Theory}



\clearpage

\appendix
\section{Conventions\label{convention}}
We use a mostly-plus metric, $\eta_{\mu\nu}=\text{diag}(-1,+1,+1,+1)$ and define
\begin{align}
    \left(\sigma^{\mu}\right)\equiv\left(1,\sigma^i\right)_{a\dot{b}},\quad\quad\left(\bar{\sigma}^\mu\right)\equiv\left(1,-\sigma^i\right)^{\dot{a}b},
\end{align}
with Pauli matrices
\begin{align}
    \sigma^{1}=\begin{pmatrix}
        0 & 1 \\
        1 & 0
    \end{pmatrix},\quad
    \sigma^{2}=\begin{pmatrix}
        0 & -i \\
        i & 0
    \end{pmatrix},\quad
    \sigma^{3}=\begin{pmatrix}
        1 & 0  \\
        0 & -1
    \end{pmatrix}.
\end{align}
Two-index spinor indices are raised/lowered using
\begin{align}
    \epsilon^{ab}=\epsilon^{\dot{a}\dot{b}}=\begin{pmatrix}
        0  & 1 \\
        -1 & 0
    \end{pmatrix}=-\epsilon_{ab}=-\epsilon_{\dot{a}\dot{b}},
\end{align}
which obey $\epsilon_{ab}\epsilon^{bc}=\delta^{c}_{a}$.

Define $\gamma$-matrices:
\begin{align}
    \gamma^\mu=\begin{pmatrix}
        0                                          & \left(\sigma^\mu\right)_{a\dot{b}} \\
        \left(\bar{\sigma}^{\mu}\right)^{\dot{a}b} & 0
    \end{pmatrix},\quad\quad
    \left\{\gamma^\mu,\gamma^\nu\right\}=-2\eta^{\mu\nu},
\end{align}
and
\begin{align}
    \gamma_5\equiv i\gamma^0\gamma^1\gamma^2\gamma^3,\quad\quad P_L=\frac{1}{2}(1-\gamma_5),\quad\quad P_R=\frac{1}{2}(1+\gamma_5).
\end{align}





\cleardoublepage
\phantomsection
\addcontentsline{toc}{section}{References}
\bibliographystyle{jhep}
\bibliography{ref}



\end{document}